\documentclass[a4paper, 16pt, oneside]{book}
\usepackage[utf8]{inputenc}
\usepackage[T2A]{fontenc}
\usepackage[english, russian]{babel}
\usepackage{indentfirst}
\usepackage{cmap}
\usepackage[unicode, hidelinks]{hyperref}
\usepackage{amsmath}
\usepackage{amsfonts}
\usepackage[indent, blackqed]{nccthm}
\usepackage{graphicx}
\usepackage{tikz}

\countstyle{section}
\ProofStyleParameters{\bfseries}{Доказательство.}
\newtheorem{Definition}{Определения}[theorem]
\newtheorem{Note}{Замечание}[theorem]
\newtheorem{Theorem}{Теорема}[theorem]
\newtheorem{Lemma}{Лемма}[theorem]

\graphicspath{ {images/} }

%\usepackage{subfiles}

\title{ВВЕДЕНИЕ В ТЕОРИЮ МНОЖЕСТВ}
\date{}

\begin{document}
\maketitle{}

\tableofcontents{}
\clearpage

\section*{Введение}
В этой книге изложены основные понятия теории множеств, которые нужны для понимания любого другого раздела математики.
В то же время не все формулировки и утверждения здесь будут точными и строгими, как это принято, например, в математической логике.
\clearpage

\chapter{Основные понятия теории множеств}

\section{Понятие множества и основные операции над множествами}

\subsection{Основные определения}

\par В математике понятие множество является неопределяемым.
Но для его понимания на данном этапе можно привести следующее определение: множество "--- это совокупность каких-либо объектов. Например, множество учеников в классе или множество книг в библиотеке.
\par Объекты, которые входят в множество, называются его элементами. Например, конкретный ученик в классе или конкретная книга в библиотеке.
\par Для обозначения множеств используются заглавные латинские буквы \(A, B, C\)\ldots,
а для обозначения элементов множеств используются строчные латинские буквы \(a, b, c\)\ldots.

\par Примеры множеств в математике\footnote{Это общепринятые обозначения, поэтому их следует запомнить.}:
\begin{itemize}
    \item \(\mathbb{N}\) "--- множество натуральных чисел
    \item \(\mathbb{Z}\) "--- множество целых чисел
    \item \(\mathbb{Q}\) "--- множество рациональных чисел
    \item \(\mathbb{I}\) "--- множество иррациональных чисел
    \item \(\mathbb{R}\) "--- множество действительных чисел
\end{itemize}

\par Есть несколько записей для описания множеств:
\begin{itemize}
    \item Перечисление элементов. Например, \(\{0, 1, 2\}\) "--- множество, состоящее из чисел 0, 1 и 2.
    \item Описание через какое-либо условие.
        Например, \(\{x \bigm| \left|x\right| < 2, x \in \mathbb{R}\}\) "--- множество действительных чисел, модуль которых меньше 2.
        Эта запись строится следующим образом. Сначала пишется одна или несколько переменных, потом вертикальная черта,
        а затем условие которому должны удовлетворять переменные.
        В записи выше \(x\) "--- это переменная, а \(x \in \mathbb{R}\) и \(\left| x \right| < 2\) - условия.
\end{itemize}

\par Первую запись удобно использовать когда множество содержит немного элементов (обычно до 5-6).
А второй запись удобно когда есть какое-то заданная условия для множества.

\begin{Definition}[Равенства множеств]
    \label{definiton:equal_sets}
    Множество A и B равны тогда и только тогда когда они состоят из одни и тех элементов.
\end{Definition}

\par Очень важную роль в теории множеств играют \textit{пустое множество}.
\begin{Definition}
    \label{definiton:emtyset}
    Пустое множество "--- это множество который не содержит никаких элементов и обозначается как \(\emptyset\)
\end{Definition}
\par Если множество представить как коробочка, то пустое множество "--- это пустая коробочка.

\par Ещё одна важное понятия это \textit{подмножество}.
\begin{Definition}
    \label{definiton:subset}
    Подмножество A является подмножеством B тогда и только тогда, когда любой элемент A является элементом B и обозначается как \(A \subseteq B\).
\end{Definition}

\begin{Definition}
    \label{definiton:strict_subset}
    Подмножество \(A\) является собственным подмножеством B тогда и только тогда,
    когда любой элемент A является элементом \(B\), но \(A\) не равно \(B\) и обозначается как \(A \subset B\).
\end{Definition}

\begin{Note}
    \label{note:emptyset_is_subset_of_any_set}
    Пустое множество является подмножеством любого множество иными словами для любого множество \(X\) верно \(\emptyset \subseteq X\).
\end{Note}

\par В математике часто встречается такое понятия как множество всех подмножеств множества \(A\).
\begin{Definition}
    \label{definiton:set_of_all_subsets}
    Множество всех подмножеств множества \(A\) это такое множества,
    что \(x\) принадлежит этому множеству тогда и только тогда когда \(x \subseteq A\).
    Обозначается это как \(2^A\), либо \(\mathcal{P(A)}\).
\end{Definition}

\subsection{Операции над множествами}
\par \textbf{Диаграмма Эйлера-Венна (иногда диаграмма Эйлера)} это графическое отображение множеств в виде кругов.
Часто используется для иллюстрации операции над множествами.

\begin{Definition}
    \label{definition:union}
    Объединение множество A и B называется такое множество, что любой элемент этого множество принадлежит хотя бы один из множеств A и B.
    Объединение множеств обозначается через \(A \cup B\)
\end{Definition}

\par \textbf{Рисунок \ref{picture:set:union}} иллюстрирует объединение множеств.

\begin{figure}
    \centering
    \begin{tikzpicture}
        \filldraw[color=blue!40!green!40!white, draw=black, thick] (3, 3) circle (1.5cm) node[anchor=east, black] {\(A\)} (5, 3) circle (1.5cm) node[anchor=west, black] {\(B\)};
        \draw (4.2, 1) node {{\Large \(A \cup B\)}};
    \end{tikzpicture}
    \caption{Объединение множеств}
    \label{picture:set:union}
\end{figure}

\par Примеры:
\begin{itemize}
    \item Если \(A = \{0, 1, 2\}\) и \(B = \{2, 3, 4\}\) то \(A \cup B = \{0, 1, 2, 3, 4\}\).
    \item Если \(A = \{\)множество положительных чисел\(\}\) и \(B = \{\)множество отрицательных чисел\(\}\), тогда \(\mathbb{Z} = A \cup \{0\} \cup B\).
    \item \(\mathbb{R} = \mathbb{Q} \cup \mathbb{I}\).
\end{itemize}

\begin{Definition}
    \label{definiton:intersection}
    Пересечение множество A и B называется такое множество, что любой элемент этого множество принадлежит обе множество одновременно.
    Пересечение множеств обозначается через \(A \cap B\).
\end{Definition}

\par \textbf{Рисунок \ref{picture:set:intersection}} иллюстрирует пересечение множеств.
\begin{figure}
    \centering
    \begin{tikzpicture}
        \draw[draw=black, thick] (3, 3) circle (1.5cm) node[anchor=east, black] {\(A\)};
        \draw[draw=black, thick] (5, 3) circle (1.5cm) node[anchor=west, black] {\(B\)};
        \begin{scope}
            \clip (3, 3) circle (1.5cm);
            \fill[color=blue!40!green!40!white, draw=black] (5, 3) circle (1.5cm);
        \end{scope}
        \draw (4.2, 1) node {{\Large \(A \cap B\)}};
    \end{tikzpicture}
    \caption{Пересечение множеств}
    \label{picture:set:intersection}
\end{figure}

\par Примеры:
\begin{itemize}
    \item Если \(A = \{0, 1, 2\}\) и \(B = \{2, 3, 4\}\) то \(A \cap B = \{2\}\).
    \item Если \(A = \{\)множество положительных чисел\(\}\) и \(B = \{\)множество отрицательных чисел\(\}\), тогда \(\emptyset = A \cap B\).
\end{itemize}

\begin{Definition}
    \label{definiton:difference}
    Разностью множество A и B называется такое множество, что любой элемент этого множество содержится в \(A\), но не содержится в \(B\).
    Разность множеств A и B обозначается как \(A \setminus B\).
\end{Definition}

\par \textbf{Рисунок \ref{picture:set:difference}} иллюстрирует разность множеств.
\begin{figure}
    \centering
    \begin{tikzpicture}
        \draw[draw=black, thick] (3, 3) circle (1.5cm) node[anchor=east, black] {\(A\)};
        \draw[draw=black, thick] (5, 3) circle (1.5cm) node[anchor=west, black] {\(B\)};
        \begin{scope}
            \clip (1.5, 1.5) rectangle (4.5, 4.5) (5, 3) circle (1.5cm);
            \fill[color=blue!40!green!40!white, draw=black] (3, 3) circle (1.5cm) node[anchor=east, black] {\(A\)};
        \end{scope}
        \draw (4.2, 1) node {{\Large \(A \setminus B\)}};
    \end{tikzpicture}
    \caption{Разность множеств}
    \label{picture:set:difference}
\end{figure}

\par Примеры:
\begin{itemize}
    \item Если \(A = \{0, 1, 2\}\) и \(B = \{2, 3, 4\}\) то \(A \setminus B = \{0, 1\}\).
    \item \(\mathbb{I} = \mathbb{R} \setminus \mathbb{Q}\).
\end{itemize}

\par Есть ещё одно операция похожая на разность множеств.
\begin{Definition}
    \label{definiton:symmetric_difference}
    Симметрическая разность множеств A и B называется такое множество, элементы которых принадлежит либо только к \(A\),
    либо только к \(B\), но не обоим одновременно и обозначается как \(A \bigtriangleup B\).
\end{Definition}

\par \textbf{Рисунок \ref{picture:set:symmetric_difference}} иллюстрирует симметрическая разность множеств.
\begin{figure}
    \centering
    \begin{tikzpicture}
        \filldraw[even odd rule, color=blue!40!green!40!white, draw=black, thick] (3, 3) circle (1.5cm) node[anchor=east, black] {\(A\)} (5, 3) circle (1.5cm) node[anchor=west, black] {\(B\)};
        \draw (4.2, 1) node {{\Large \(A \bigtriangleup B\)}};
    \end{tikzpicture}
    \caption{Симметрическая разность множеств}
    \label{picture:set:symmetric_difference}
\end{figure}

\par Примеры:
\begin{itemize}
    \item Если \(A = \{0, 1, 2\}\) и \(B = \{2, 3, 4\}\) то \(A \bigtriangleup B = \{0, 1, 3, 4\}\).
    \item Если \(A = \{\)множество натуральных чисел делимое на 2\(\}\), \(B = \{\)множество натуральных чисел делимое на 3\(\}\), то
        \(A \bigtriangleup B = \{\)множество натуральных чисел либо делимое на 2, либо делимое на 3, но не делимое на 6 \(\}\).
\end{itemize}

%рисунок 1.4

\subsection{Кванторы}
\par В математике часто используется такие понятия как <<кванторы>>. Здесь не будет дано строгие определения квантора, а просто объясняется как их расшифровать.
Кванторы используется в различных утверждениях где есть какие-то переменные величины, чтобы показать, именно каким образом выполняется утверждения.
\begin{itemize}
    \item \(\forall x\) "--- неограниченный квантор общности. \((\forall x)\Phi\) означает <<для любого \(x\) выполняется утверждения \(\Phi\)>>.
    \item \(\exists x\) "--- неограниченный квантор существование. \((\exists x)\Phi\) означает <<существует такой \(x\), такое что для него верно \(\Phi\)>>.
    \item \(\forall x \in A\) "--- ограниченный квантор общности. \((\forall x \in A)\Phi\) означает <<для любого \(x\) из \(A\) выполняется утверждения \(\Phi\)>>.
        Различия ограниченного и неограниченного квантора в том, ограниченный квантор говорит верности утверждения для конкретного множества \(A\).
        То есть не какой-то произвольный объект, а элемент конкретного множества.
    \item \(\exists x \in A\) "--- ограниченный квантор существование. \((\exists x \in A)\Phi\) означает <<существует такой \(x\) в \(A\),
        такое что для него верно \(\Phi\)>>. Разница неограниченного и ограниченного квантора существование такая же.
\end{itemize}
\par Если вы первый раз слышите о кванторах, то вам возможно в начале сложно будет их расшифровать.
Но это надо уметь. Потому без кванторов нельзя написать сложные утверждения компактно и почти всех литературах они есть.
И эти учебники тоже не исключения.
\begin{Note}
    \par В многих литературах и учебниках если речь идёт о конкретном множестве то часто вместо ограниченного квантора используется неограниченный квантор.
    При этом подразумевается что это ограниченный квантор. Например, в книгах о математическом анализе речь почти всегда идёт о действительных числах.
\end{Note}
То есть функции от действительных чисел, свойство действительных чисел и т.д. и в таком случае не пишут \(\forall x \in \mathbb{R}\),
а просто \(\forall x\). И часто кванторы пишется без скобки.
\par Неограниченные кванторы обычно используется в своём смысле в книгах о математической логике и теории множеств.
\par Теперь давайте рассмотрим некоторые примеры чтобы лучше понимать как расшифровать кванторы.
\begin{itemize}
    \item <<Для любого натурального числа \(m\) и \(n\), верно то что \(m + n\) тоже натуральное>>. То же самое можно писать так через кванторы: \par
        \begin{equation*}
            (\forall m \in \mathbb{N})(\forall n \in \mathbb{N})(m + n \in \mathbb{N})
        \end{equation*}
    \item <<Для любого положительного \(\epsilon\), существует такой натуральный \(N\),
        такое что для любого \(n\) больше чем \(N\) верно \(\left|a_n - b\right| < \epsilon\) (определения предела последовательности)>>.
        То же самое через кванторы: \par
        \begin{equation*}
            (\forall \epsilon \in R \land \epsilon > 0)(\exists N \in \mathbb{N})(\forall n \in \mathbb{N} \land n > N) (\left|a_n - b\right| < \epsilon)
        \end{equation*}
\end{itemize}

\subsection{Семейства множеств}

\par В теории множеств есть такая понятие как \textit{Семейства множеств}.

\par Есть ещё одно операция похожая на разность множеств.
\begin{Definition}
    \label{definiton:set_families}
    Пусть у нас дано множества \(I\). Пусть для любого \(i\) из \(I\) существует \(A_i\).
    Семейством множеств мы называем совокупность множеств \(A_i\ |\ i \in I\).
\end{Definition}

\begin{Definition}
    \label{definiton:union_set_families}
    Объединение семейства множеств \((A_i)_{i \in I}\) это множество
    (обозначаемое \(\underset{i \in I}{\bigcup} A_i\)) для любого элемента \(a \in \underset{i \in I}{\bigcup} A_i\)
    существует \(i \in I\), такое что \(a \in A_i\)
\end{Definition}

\begin{Definition}
    \label{definiton:intersect_set_families}
    Пересечение семейства множеств \((A_i)_{i \in I}\) это множество
    (обозначаемое \(\underset{i \in I}{\bigcap} A_i\)) для любого элемента \(a \in \underset{i \in I}{\bigcap} A_i\)
    и для любого \(i \in I\), верно \(a \in A_i\)
\end{Definition}

\section{Декартовое произведения множеств. Бинарное и \(n\)-арное отношение. Отношения эквивалентности и порядка}

\subsection {Множество из \(n\) элементов и упорядоченной \(n\)-ок}
\par Нужно заметить что множество сам по себе не даёт никакой порядок для своих элементов.
Мы можем писать \(\{1, 2, 3\}\), \(\{2, 1, 3\}\), \(\{3, 2, 1\}\), но все они просто разные обозначение одного множество.
\par В математике (и даже в жизни), часто нужно какая-то совокупность элементов для которых порядок элементов важен.
Например, в декартовом системы координат (2, 3) и (3, 2) это разные объекты. И сейчас речь будет идти о том как определять такие объекты через множества.

\begin{Definition}
    \label{definiton:ordered_pair}
    Упорядоченной парой с первым элементом \(x\) и вторым элементом \(y\) называется множество \(\{\{x\}, \{x, y\}\}\)
    и обозначается как \((x, y)\) (или как \(\langle x, y\rangle\)).
\end{Definition}

\begin{Definition}
    \label{definiton:orderen_3}
    Для трёх элементов \(x\), \(y\), \(z\) упорядоченной тройкой является множество \(((x, y), z)\) (или \(\langle\langle x, y \rangle, z\rangle\)).
    Это множество обозначается как \((x, y, z)\) (или \(\langle x, y, z \rangle\)).
\end{Definition}

\par Таким же образом можно определить упорядоченных \(n\)-ок.

\begin{Definition}
    \label{definiton:orderen_n}
    Для \(n\) элементов \(a_1\), \(a_2\), \ldots \(a_n\) упорядоченной \(n\)-кой является множество \(((a_1, a_2, \ldots, a_{n - 1}), a_n)\)
    (или \(\langle\langle a_1, a_2, \ldots, a_{n - 1}\rangle, a_n \rangle\)).
    Это множество обозначается как \((a_1, a_2, \ldots, a_n)\) (или \(\langle a_1, a_2, \ldots, a_n \rangle\)).
\end{Definition}

\par Давайте для понимания распишем упорядоченную тройку \((a_1, a_2, a_3)\):
\begin{multline*}
    (a_1, a_2, a_3) = ((a_1, a_2), a_3) = \Big\{\big\{(a_1, a_2)\big\}, \big\{(a_1, a_2), a_3\big\}\Big\} = \\ = \Bigg\{\bigg\{\Big\{\{a_1\}, \big\{a_1, a_2\big\}\Big\}\bigg\}, \bigg\{\Big\{\big\{a_1\big\}, \big\{a_1, a_2\big\}\Big\}, a_3\bigg\}\Bigg\}
\end{multline*}

\subsection{Бинарное отношение}

\par И так мы определили что такое упорядоченная пара. И теперь определим что такое множество упорядоченных пар.

\begin{Definition}
    \label{definiton:binary_relation}
    Бинарное отношение (или соответствие) называется множество, которое каждый элемент является упорядоченной парой.
\end{Definition}

\par Если для \(x\) и \(y\) верно, то что \((x, y) \in R\) (\(R\) - бинарное отношение), то говорится \(x\) и \(y\) находится в отношении \(R\)
и обозначается как \(xRy\).

\par Примером бинарного отношения может быть отношение меньше или равно (то есть \(\le\))

\par Часто бывает интересно отношение в определённых множествах. Но прежде чем продолжить мы должны определить что такое \textit{декартовое произведение}.

\begin{Definition}
    \label{definiton:cartesian_product}
    Декартовое произведение множеств \(A\) и \(B\) (обозначаемое через \(A \times B\)) это множества таких упорядоченных пар что выполняется два условия:
    \begin{enumerate}
        \item Для любого \(a\) из \(A\) и \(b\) из \(B\) верно то что \((a, b)\) принадлежит \(A \times B\)
            (то есть \((\forall a \in A)(\forall b \in B)[(a, b) \in A \times B]\))
        \item Для любого \((a, b) \in A \times B\) верно то что \(a \in A\) и \(b \in B\)
            (то есть \((\forall (a, b) \in A \times B)[a \in A \land b \in B]\))
    \end{enumerate}
\end{Definition}

\par Из определения следует, то что \(A \times B\) состоит только и только из таких упорядоченных пар,
в которых первый элемент это элемент \(A\), а второй элемент это элемент \(B\).

\begin{Definition}
    \label{definiton:binary_relation_in_A_B}
    Бинарное отношение заданное в множествах \(A\) и \(B\) "--- это бинарное отношение,
    которое для каждого элемента \((a, b)\) верно то что \(a \in A\) и \(b \in B\).
\end{Definition}

\par Из определения выше следует что если \(R\) бинарное множество заданное в множествах \(A\) и \(B\)
то верно \(R \subseteq A \times B\).

\begin{Definition}
    \label{definiton:domain}
    Областью определение бинарного отношение \(R\) называется множество \(\{x \bigm| \exists y (x, y) \in R\}\) и обозначается как \(D(R)\).
\end{Definition}

\begin{Definition}
    \label{definiton:range}
    Областью значение бинарного отношение \(R\) называется множество \(\{y \bigm| \exists x (x, y) \in R\}\) и обозначается как \(E(R)\).
\end{Definition}

\begin{Definition}
    \label{definiton:image_of_element}
    Образ элемента \(x\) по отношению \(R\) называется такое множество \(\{y \bigm| (x, y) \in R\}\).
\end{Definition}

\begin{Definition}
    \label{definiton:prototype}
    Прообраз элемента \(y\) по отношению \(R\) называется такое множество \(\{x \bigm| (x, y) \in R\}\)
\end{Definition}

\par Аналогичном образом определяется \(n\)-арное отношение, декартовое произведение n множеств
и \(n\)-арное отношение в множествах \(A_1, A_2 \ldots A_n\).

\section{Отображение. Инъективные и сюръективные отображение}

\par Отображения (или функции) является самым важным типом бинарных отношений в математике

\begin{Definition}
    \label{definiton:function}
    Бинарное отношение \(f\) называется отображением тогда и только тогда, когда для любого  \(y_1\) и \(y_2\) если \((x, y_1) \in f\) и \((x, y_2) \in f\), то \(y_1 = y_2\).
\end{Definition}

\par То есть в отображениях не бывает две разных второго элемента для одного первого элемента.
Как и в случае для бинарных отношений наибольший интерес представляет такие отображения которые заданы в конкретных множествах.

\begin{Definition}
    \label{definiton:function_in_sets}
    Бинарное отношение \(f \in A \times B\) называется отображением множества \(A\) в множестве \(B\) тогда и только тогда,
    когда во первых для любого \(x \in A\) существует \(y \in B\) такое что \((x, y) \in f\),
    во вторых для любого \(y_1 \in B\) и \(y_2 \in B\) если \((x, y_1) \in f\) и \((x, y_2) \in f\), то \(y_1 = y_2\).
    Отображения \(f\) от \(A\) в \(B\) обозначается как \(f: A \rightarrow B\).
\end{Definition}

\par Примеры:
\begin{itemize}
    \item Элементарные функции вроде \({y = \sin{x}\ (f: \mathbb{R} \rightarrow \mathbb{R}),\ y = a ^ x\ (f: \mathbb{R_{+}} \rightarrow \mathbb{R})}\)
    \item Функция от натурального числа который возвращает 1 для нечётных чисел и возвращает 2 для чётных чисел тоже является функцией 
    \({f: \mathbb{N} \rightarrow \mathbb{N}}\)
\end{itemize}

\par Некоторые важные понятия относящие к отображениям:
\begin{Definition}
	\label{definition:function_domain}
	Область определения отображения \(f\) является множества \(X = \{x \bigm| (x, y) \in f\}\). Обозначается через \(D(f)\).
\end{Definition}

\begin{Note}
	\label{note:A_is_equal_D(f)}
	Для отображения \(f\) заданных в множествах \(A\) и \(B\) верно то что \(D(f) = A\)
\end{Note}

\begin{Definition}
	\label{definition:function_range_of_values}
	Область значения отображения \(f\) является множества \(Y = \{y \bigm| (x, y) \in f\}\). Обозначается через \(E(f)\).
\end{Definition}

\begin{Definition}
	\label{definition:image_of_element}
	Образом элемента \(x \in D(f)\) называется \(y\) такое что \((x, y) \in f\) и обозначается как \(f(x)\).
\end{Definition}

\begin{Definition}
	\label{definition:prototype}
	Прообразом элемента \(y \in E(f)\) называется множества \(X = \{x \bigm| (x, y) \in f\}\) и обозначается как \(f^{-1}(y)\).
\end{Definition}

\begin{Note}
	\label{note:prototype_is_set}
	Прообраз какого-то элемента вообще говоря является множеством. 
	Например: \({f(x)=\sin{x}}\), тогда \linebreak \({f^{-1}(1) = \{\frac{\pi}{2} + 2 \pi k \bigm| k \in \mathbb{Z}\}}\)
\end{Note}

\begin{Definition}
	\label{definition:image_of_set}
	Образом множества \(X\) называется множества \({f(X) = \{y \bigm| (x, y) \in f, x \in X\}}\)
\end{Definition}

\begin{Definition}
	\label{definition:prototype_of_set}
	Прообразом множества \(Y\) называется множества \({f^{-1}(Y) = \{x \bigm| (x, y) \in f, y \in Y\}}\)
\end{Definition}

\par Если обратим внимание на \textbf{замечании \ref{note:prototype_is_set}}, 
то понимаем что в общем случае функция могут отображать разные аргументы на разные значения.
В то же время существует функции которые, отображает разные аргументы на разные значения (например: \({f(x)=x}\)).
Следующая определения об этом.

\begin{Definition}
	\label{definition:injective}
	Функция \(f\) называется инъективным, если \linebreak \({x_1 \neq x_2 \Rightarrow f(x_1) \neq f(x_2)}\). 
\end{Definition}   

\begin{Definition}
	\label{definition:surjective}
	Функция \(f: A \rightarrow B\) называется сюръективным, если \({f(A) = B}\).
\end{Definition}

\section{Мощность множества. Счётные и континуальные множества}

\section{Отношения эквивалентности и порядка}

\subsection{Свойства бинарных отношений}

\par Часто интересно изучить бинарное отношение который является подмножеством \(A \times A\)
и такие отношение называется \textit{бинарное отношение заданное в множестве \(A\)}.

\par Есть различные свойства которые может обладать такие отношение(\(R\) это бинарное отношения заданное \(A\)):
\begin{itemize}
	\item \textit{Рефлексивность}. Отношения рефлексивно тогда и только тогда, когда \((\forall a in A)((a, a) \in R)\).
	Часто множества \(\{(a, a) \bigm| a \in A\}\) называется \textit{диагоналом множества \(A\)} и обозначается \(id_A\).
	Учитывая это можно определить рефлексивное отображения таким образом \(id_A \subseteq R\).
	\item \textit{Антирефлексивность}. Отношения антирефлексивно тогда и только тогда, когда \((\forall a \in A)((a, a) \notin R)\).
	Антирефлексивность можно определять и так \(id_A \cap R = \emptyset\).
	\item \textit{Симметричность}. Отношение симметрично тогда и только тогда, когда \((\forall x,y \in A)[(x, y) \in R \Rightarrow (y, x) \in R]\)
	\item \textit{Антисимметричность}. Отношение антисимметрично тогда и только тогда, когда \((\forall x,y \in A)[(x, y) \in R \land (y, x) \in R \Rightarrow (x = y)]\)
	\item \textit{Транзитивность}. Отношение транзитивности тогда и только тогда, когда \((\forall x,y,z \in A)[(x, y) \in R \land (y, z) \in R \Rightarrow (x, z) \in R]\)
	\item \textit{Линейность}. Отношение линейно тогда и только тогда, когда \((\forall x,y \in A)[(x, y) \in R \lor (y, x) \in R]\). Это означает что для любые две элементы находится в отношении \(R\)
\end{itemize}

\par Среди отношений в данном множество очень интересны так называемые \textit{отношения эквивалентности} и \textit{отношения порядка}

\subsection{Отношения эквивалентности}

\begin{Definition}
	\label{definiton:equivalence_relation}
	\(R\) является отношениям эквивалентности тогда и только тогда, когда она рефлексивно, симметрично и транзитивно
\end{Definition}

\par С помощью отношении эквивалентности можно каждому элементу множества сопоставить некоторое подмножества \(A\),
который называется \textit{классом эквивалентности} этого элемента.

\begin{Definition}
	\label{definiton:equivalence_class}
	Классом эквивалентности \(a \in A\) по отношении \(R\) называется такое множество \(\{x \bigm| (a, x) \in R\}\)
	и обозначается как \(\left|a\right|\)
\end{Definition}

\begin{Note}
	\label{note:equivalence_class_reflection}
	Для любого отношения эквивалентности в силу рефлексивности верно \(a \in \left|a\right|\).
\end{Note}

\begin{Note}
	\label{note:equivalence_class_symmetric}
	Классу эквивалентности \(a\) относится те \(x\) для которых \((x, a) \in R\) тоже.
	Поскольку в силу симметричности отношения эквивалентности \((x, a) \in R \Rightarrow (a ,x) \in R\).
\end{Note}

\begin{Theorem}
	\label{theorem:equivalence_class_of_two_elements}
	Пусть задан множества \(A\) и бинарное отношение \(R\) в нём.
	Для любого \(a\) и \(b\) либо \(\left|a\right|=\left|b\right|\) совпадает,
	либо \(\left|a\right| \cap \left|b\right| = \emptyset\)
	\proof
	Пусть \(\left|a\right| \cap \left|b\right| \ne \emptyset\) и
	пусть \(c \in \left|a\right| \cap \left|b\right|\).
	Мы должны доказать \(d \in \left|a\right| \Leftrightarrow d \in \left|b\right|\).
	Мы докажем для одну сторону. Пусть \(d \in \left|a\right|\).
	Тогда \((d, a) \in R\) и \((a, c) \in R\)(в силу того \(c \in \left|a\right|\)),
	от этого следует \((d, c) \in R\).
	Поскольку \(c \in \left|b\right|\), от этого следует \((c, b) \in R\).
	В силу транзитивности \(R\) \(((d, c) \in R) \land ((c, b) \in R) \Rightarrow ((d, b) \in R) \Leftrightarrow (d \in \left|b\right|)\). 
	В другую сторону доказывается аналогично.
	\qedsymbol
\end{Theorem}

\par Теперь определим так называемую \textit{разбиение множеств}
\begin{Definition}
	\label{definiton:set_splitting}
	Пусть задан множества \(A\) и семейства множеств \((A_i)_{i \in I}\) такое что \((\forall i \in I)(A_i \subseteq A); (\forall i,j \in I)(i \ne j \Rightarrow A_i \cap A_j \ne \emptyset); (\underset{i \in I}{\bigcup} = A)\). Такая семейства множеств называется разбиение множеств.
\end{Definition}

От \textbf{теоремы \ref{theorem:equivalence_class_of_two_elements}} следует что классы эквивалентности есть разбиение класса.
Из-за этого можно сказать \textit{отношение эквивалентности \(R\) разбивает множества \(A\)}.
Следующая теорема показывает есть однозначное соответствие между разбиением множества и
эквивалентным отношением в данном множестве.
\begin{Theorem}
	\label{theorem:splitting_by_equivalence_relation}
	В множестве \(A\) для любая бинарная отношения эквивалентности разбивает множества \(A\)
	и для любого разбиении множества \(A\) можно найти соответствующие разбиение.
	\proof Первая утверждения непосредственно вытекает из \textbf{теоремы \ref{theorem:equivalence_class_of_two_elements}} и \textbf{определении \ref{definiton:set_splitting}}.
	\par Докажем второе утверждения. Пусть у нас есть разбиение \((A_i)_{i \in I}\).
	Построим отношение \(R\) таким образом:
	\((\forall x,y \in A)[(\exists i \in I)((x, y) \in R \Leftrightarrow (x \in A_i \land y \in A_i))]\).
	Докажем что это бинарное отношение эквивалентная.
	\begin{enumerate}
		\item Рефлексивность. Поскольку при объединении разбиение даёт вся множества \(A\), то для любого\(x\)
		можно найти \(i \in I\) такое что \(x \in A_i\). От этого и определении самого отношения \(R\) следует
		что \((x, x) \in R\)
		\item Симметричность. \((x, y) \in R\) означает что \(y\) и \(x\) является элементами одной части разбиение.
		От этого следует то что \((y, x) \in R\)
		\item Транзитивность. \(((x, y) \in R \land (y, z) \in R) \Rightarrow (\exists i)(x, y, z \in A_i)
		\Rightarrow ((x, z) \in R)\)
	\end{enumerate}
	\qedsymbol
\end{Theorem}

\subsection{Отношения порядка}

\par В теории множеств не менее важно так называемые отношении порядка.
Если не строго говорить это такие бинарные отношение которые ведёт себя похожим образом на отношения сравнения для чисел.
И есть несколько типов отношении порядка.

\begin{itemize}
	\item \textbf{Отношения (нестрогого) предпорядка} "--- это рефлексивное и транзитивное отношение
	\item \textbf{Отношения строгого предпорядка} "--- это антирефлексивное и транзитивное отношение
	\item \textbf{Отношения (нестрогого) порядка} "--- это рефлексивное, антисимметричное и транзитивное отношение
	\item \textbf{Отношения строгого порядка} "--- это антирефлексивное, антисимметричное и транзитивное отношение
	\item \textbf{Отношения линейного (нестрогого) порядка} "--- это рефлексивное, антисимметричное, транзитивное и линейное отношение
	\item \textbf{Отношения линейного строго порядка} "--- это рефлексивное, антисимметричное, транзитивное и линейное отношение
\end{itemize}

\end{document}
